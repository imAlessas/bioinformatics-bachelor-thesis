%!TEX root = ../main.tex

\chapter{Introduzione}\label{chp:introduction}

Ad oggi l'avanzamento della genomica — branca della biologia molecolare che si occupa di studiare il genoma degli esseri viventi — si è rivelato notevolmente significativo al fine di approfondire e comprendere malattie legate alle mutazioni del genoma degli individui. Si stima che solamente una percentuale tra l'1\% e il 2\% del DNA contiene i \textsl{geni}, ovvero particolari regioni che contengono tutte le informazioni necessarie per la sintesi degli aminoacidi che poi comporranno le proteine\,\cite{sahu2011identification, pollard2022cell}. Ciò nonostante, la quasi totalità dei disturbi genomici è dovuta alle mutazioni nelle regioni non codificanti\,\cite{zhang2015non} — dette \textsl{varianti non codificanti}. Le mutazioni in queste zone del genoma, che apparentemente svolgono funzioni marginali, sono responsabili dello sviluppo di disturbi importanti, come le \textsl{malattie mendeliane}\footnote{Le malattie mendeliane, causate dalla mutazione di un singolo gene, includono la fibrosi cistica e il morbo di Huntington.}\,\cite{french2020role, chial2008mendelian}, l'epilessia\,\cite{pagni2022non}, malattie cardiovascolari\,\cite{kapoor2014enhancer, zhang2015non} e soprattutto tumori — tra cui il cancro del colon-retto e il tumore al seno\,\cite{khurana2016role, tian2019systematic, bojesen2013multiple, michailidou2017association}.Risulta quindi vitale continuare a studiare gli effetti che le varianti non codificanti in sequenze genomiche hanno sugli individui.

Proprio a questo proposito, con l'avvento dell'intelligenza artificiale, in particolare del \textsl{deep learning}, si continuano a trovare e perfezionare soluzioni informatiche che permettano di delineare con sempre maggior precisione il ruolo che hanno le mutazioni nelle regioni non codificanti del DNA.\@ Grazie a queste nuove tecnologie, la \textsl{genomica funzionale} — area della genomica che si interessa a descrivere le relazioni che ci sono tra i componenti di un sistema biologico, come geni e proteine\,\cite{caudai2021ai} — ha avuto un forte impulso nell'approfondire le varianti non codificanti, tuttavia rimangono ancora significative lacune nella comprensione della relazione tra mutazioni genetiche ed espressione genica. L'utilizzo di tecniche di deep learning risulta quindi cruciale per continuare la ricerca; a questo proposito, nel presente elaborato accademico verranno descritti e paragonati tre \textsl{tool} che utilizzano le \textsl{reti neurali convoluzionali} per predire l'effetto delle varianti non codificanti su sequenze genomiche: DeepSEA\,\cite{zhou2015predicting}, Basset\,\cite{kelley2016basset} e DeepSATA\,\cite{ma2023deepsata}.

Più precisamente, il Capitolo\,\ref{chp:biological-background} introdurrà le basi della biologia molecolare, necessarie per comprendere interamente l'importanza delle varianti non codificanti. Successivamente, nel Capitolo\,\ref{chp:neural-networks} saranno approfonditi i principi fondamentali delle reti neurali e di come le reti convoluzionali possono essere utilizzate come ottimo strumento per predire l'effetto di sequenze genomiche. Il Capitolo\,\ref{chp:CNN-non-coding-variants} invece esaminerà i dettagli implementativi di ciascuno dei tre tool, indagando principalmente negli aspetti legati alla codifica delle sequenze, alla struttura della rete e al \textsl{dataset} utilizzato per allenare il modello. Infine nel Capitolo\,\ref{chp:discussion} 

\todo{Ascolta registrazione e perfeziona introduzione}
\todo{Rimuovi stato dell'arte e aggiungi sequenziamento nell'introduzione}

% 

\section{Stato dell'arte}

Il continuo avanzamento di strumenti informatici e computazionali ha nettamente agevolato la comprensione di numerosi aspetti legati alla biologia favorendo la nascita di una branca che unisce informatica e biologia, la bioinformatica. Inoltre, con l'emergere dell'intelligenza artificiale, svariati strumenti bioinformatici sono stati creati come AlphaFold, ROSETTA3, Bioconductor, Clustale  DeepVirFInder\,\cite{jumper2021highly, leaver2011rosetta3, gentleman2004bioconductor, larkin2007clustal, ren2020identifying}. Oltre a questi rimangono molto diffusi anche strumenti che non hanno basi di deep learning ma sfruttano solo la potenza di calcolo delle macchine moderne, come BLAST\,\cite{altschul1990basic}. La potenza e l'accuratezza di questi nuovi strumenti basati sull'AI è stata favorita anche dall'immensità di dati biologici che sono stati raccolti nell'ultimo ventennio e mantenuti in enormi basi di dati come l'\textit{NCBI database} e il PDB\,\cite{sherry2001dbsnp, burley2017protein}

In particolare, per quanto riguarda le varianti non codificanti, lo stato dell'arte è rappresentato da DeepSEA, Basset e DeepSATA\,\cite{zhou2015predicting, kelley2016basset, ma2023deepsata}. Questi tre tool si occupano di fornire una predizione accurata sugli effetti delle varianti non codificanti attraverso le reti neurali convoluzionali.

% https://books.google.it/books?hl=it&lr=&id=Cg4WAgAAQBAJ&oi=fnd&pg=PP1&dq=Introduction+to+cell+biology&ots=yg4LdM46O3&sig=FkW8Ei_rOFccb96Sw3A28QsHuFo&redir_esc=y#v=onepage&q=Introduction%20to%20cell%20biology&f=false

% https://books.google.it/books?hl=it&lr=&id=mXiiEAAAQBAJ&oi=fnd&pg=PP1&dq=cellular+biology&ots=8O1TrOZBXp&sig=fqZhVnT0H-I3qEu1iPk4v2nvZi8&redir_esc=y#v=onepage&q=cellular%20biology&f=false

% https://books.google.it/books?hl=it&lr=&id=z4BDRcLrekMC&oi=fnd&pg=PR19&dq=cell+nucleus+organization&ots=wiXN8JbVpC&sig=oviVqMKJqsvcYwe4X37tQM28P3Y&redir_esc=y#v=onepage&q=cell%20nucleus%20organization&f=false

% Images:
% https://www.genome.gov/genetics-glossary
