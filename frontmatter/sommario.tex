\begin{abstract}[it]
    Le mutazioni non codificanti, aree del genoma che non vengono codificate in proteine, sono la principale causa dei disturbi genetici, tra cui le malattie mendeliane. Gli effetti funzionali di queste variazioni sono ancora difficili da comprendere completamente. Grazie a nuove tecnologie di sequenziamento — che hanno permesso un significativo arricchimento di banche di dati biologici — e allo sviluppo di supporto hardware sufficientemente potente è stato possibile progettare strumenti basati sulle reti neurali che analizzano sequenze genomiche e forniscono informazioni rilevanti legate agli effetti funzionali di queste particolari mutazioni genomiche.
    
    Questo elaborato ha l'obiettivo di dare un'introduzione alla biologia molecolare e fornire gli strumenti matematici per comprendere le reti neurali. Più precisamente verrà esplorato il funzionamento e la struttura di una rete neurale convoluzionale con l'obiettivo di riuscire ad analizzare tre strumenti, basati su questa tecnologia. Saranno esaminati DeepSEA, Basset e DeepSATA, tre tool che hanno lo scopo di approfondire gli effetti causati dalle mutazioni non codificanti del genoma.
\end{abstract}

% Questo elaborato mira ad approfondire il funzionamento delle reti neurali convoluzionali e di come questi modelli di deep learning siano in grado di estrarre significative informazioni da sequenze genomiche, analizzandone le zone non codificanti. In particolare, verranno comparati tre tool basati sulle CNN — DeepSEA, Basset e DeepSATA — e sarà fornita una revisione delle loro prestazioni. 
