%!TEX root = ../main.tex
\chapter{Introduzione}\label{chp:introduction}

Ad oggi l'avanzamento della genomica — branca della biologia molecolare che si occupa di studiare il genoma degli esseri viventi — si è rivelato notevolmente significativo al fine di approfondire e comprendere malattie legate alla sequenza genomica degli individui. Si stima che solamente l'1\% ~ 2\% del \ac{DNA} contiene i \textsl{geni}, ovvero particolari regioni che vengono codificate in proteine\,\cite{sahu2011identification}. Ciò nonostante, la quasi totalità dei disturbi genomici è dovuta alle mutazioni nelle regioni non codificanti\,\cite{zhang2015non} — dette \textsl{varianti non codificianti}. Le variazioni in queste regioni, che apparentemente svolgono funzioni marginali, possono causare importanti disturbi e malattie, come tumori\,\cite{khurana2016role}, 




\section{Varianti non codificanti}





\section{Stato dell'arte}




% \ac{CNN}
% \begin{displayquote}
% Lorem ipsum dolor sit amet, consectetur adipiscing elit, sed do eiusmod tempor incididunt ut labore et dolore magna aliqua.
% \end{displayquote}