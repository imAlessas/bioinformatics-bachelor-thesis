%!TEX root = ../main.tex
\begin{abstract}
    Non-coding variants, genome regions that are not translated into proteins, are a major cause of genetic diseases, such as Mendelian disorders. The functional effects of these mutations remain difficult to fully comprehend. However, thanks to advances in sequencing technologies — which have greatly enriched biological data banks — and the development of sufficiently powerful hardware, it has become possible to design neural network-based tools capable of analyzing genomic sequences and providing valuable insights into the functional effects of these specific DNA regions.

    This thesis aims to introduce molecular biology concepts and provide mathematical tools for understanding neural networks. Specifically, it will explore the structure and functioning of convolutional neural networks with the goal of analyzing three tools based on this technology. The thesis will focus on DeepSEA, Basset, and DeepSATA — three tools designed to enhance the understanding of the functional impact of non-coding variants.
\end{abstract}

% This thesis aims to deepen the understanding of the functioning of convolutional neural networks and how these deep learning models are able to extract significant information from genomic sequences, analyzing their non-coding regions. In particular, three CNN-based tools — DeepSEA, Basset and DeepSATA — will be compared and a review of their performance will be provided.
