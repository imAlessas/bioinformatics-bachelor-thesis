%!TEX root = ../main.tex
\chapter{Introduzione}\label{chp:introduction}

Ad oggi l'avanzamento della genomica — branca della biologia molecolare che si occupa di studiare il genoma degli esseri viventi — si è rivelato notevolmente significativo al fine di approfondire e comprendere malattie legate alle mutazioni del genoma degli individui. Si stima che solamente una percentuale tra l'1\% e il 2\% del DNA contiene i \textsl{geni}, ovvero particolari regioni che contengono tutte le informazioni necessarie per la sintesi degli aminoacidi che poi comporranno le proteine\,\cite{sahu2011identification}. Ciò nonostante, la quasi totalità dei disturbi genomici è dovuta alle mutazioni nelle regioni non codificanti\,\cite{zhang2015non} — dette \textsl{varianti non codificianti}. Le mutazioni in queste zone del genoma, che apparentemente svolgono funzioni marginali, sono responsabili dello sviluppo di disturbi importanti, come le \textsl{malattie mendeliane}\footnote{Le malattie mendeliane, causate dalla mutazione di un singolo gene, includono la fibrosi cistica e il morbo di Huntington.}\,\cite{french2020role, chial2008mendelian}, l'epilessia\,\cite{pagni2022non}, malattie cardiovascolari\,\cite{kapoor2014enhancer, zhang2015non} e soprattuto tumori — tra cui il cancro del colon-retto e tumore al seno\,\cite{khurana2016role, tian2019systematic, bojesen2013multiple, michailidou2017association}.

Risulta quindi vitale continuare a studiare gli effetti che le varianti non codificanti in sequenze genomiche hanno sugli individui. Proprio a questo proposito, con l'avvento dell'intelligenza artificiale, in particolare del \textsl{deep learning}, si continuano a trovare e perfezionare soluzioni che permettano di delineare sempre con più precisione il ruolo che hanno le mutazioni nelle regioni non codificanti del DNA.\@ Grazie a queste nuove tecnologie, la \textsl{genomica funzionale} — area della genomica che si interessa a descrivere le relazioni che ci sono tra i componenti di un sistema biologico, come geni e proteine\,\cite{caudai2021ai} — ha avuto un forte impulso nell'approfondire le varianti non codificanti ma rimangono ancora significative lacune nella comprensione riguardante la relazione tra mutazioni genetiche ed espressione genica. L'utilizzo di tecniche di deep learning e quindi di reti neurali rimane quindi cruciale per continuare la ricerca; a questo proposito in questo documento verranno descritti e paragonati tre \textsl{tool} che utilizzano le \textsl{reti neurali convoluzionali} per predire l'effetto delle varianti non codificanti su seguenze genomiche: DeepSEA\,\cite{zhou2015predicting}, Basset\,\cite{kelley2016basset} e DeepSATA\,\cite{ma2023deepsata}.



\section{Varianti non codificanti}





\section{Stato dell'arte}



\section*{Storia}

% \ac{CNN}
% \begin{displayquote}
% Lorem ipsum dolor sit amet, consectetur adipiscing elit, sed do eiusmod tempor incididunt ut labore et dolore magna aliqua.
% \end{displayquote}