%!TEX root = ../main.tex

\chapter{Introduzione}\label{chp:introduction}

Ad oggi l'avanzamento della genomica — ramo della biologia molecolare che si occupa di studiare il genoma degli esseri viventi — si è rivelato notevolmente significativo al fine di approfondire e comprendere malattie legate alle mutazioni del genoma degli individui. Si stima che solamente una percentuale tra l'1\% e il 2\% del \acs{DNA} contiene i \textsl{geni}, ovvero particolari regioni che contengono tutte le informazioni necessarie per la sintesi degli aminoacidi che poi comporranno le proteine\,\cite{sahu2011identification, pollard2022cell}. Ciò nonostante, la quasi totalità dei disturbi genomici è dovuta alle mutazioni nelle regioni non codificanti\,\cite{zhang2015non} — dette \textsl{varianti non codificanti}. Le mutazioni in queste zone del genoma, che apparentemente svolgono funzioni marginali, sono responsabili dello sviluppo di disturbi importanti, come le \textsl{malattie mendeliane}\footnote{Le malattie mendeliane, causate dalla mutazione di un singolo gene, includono la fibrosi cistica e il morbo di Huntington.}, l'epilessia, malattie cardiovascolari e soprattutto tumori — tra cui il cancro del colon-retto e il tumore al seno\,\cite{french2020role, chial2008mendelian, pagni2022non, kapoor2014enhancer, zhang2015non, khurana2016role, tian2019systematic, bojesen2013multiple, michailidou2017association}.Risulta quindi vitale continuare a studiare gli effetti che le varianti non codificanti in sequenze genomiche hanno sugli individui.

Negli ultimi decenni, il progredire delle tecniche di \textsl{sequenziamento}\,\cite{pareek2011sequencing} ha dato uno slancio rilevante allo sviluppo della \textsl{bioinformatica} — disciplina che unisce informatica e biologia. La bioinformatica si interessa a organizzare dati biologici in modo tale da facilitare l'accesso e l'inserimento di nuove informazioni (come il \textsl{PDB}\,\cite{burley2017protein}), sviluppare \textsl{tool} che permettono l'analisi dei dati e infine fornire una interpretazione significativa dei risultati ottenuti\,\cite{luscombe2001bioinformatics}. Più recentemente, l'accrescimento dei dati biologici e il costante avanzamento della potenza di calcolo hanno reso possibile l'applicazione di tecniche di \textsl{deep learning} (\acs{DL}) anche nel campo della bioinformatica. Questo notevole progresso consente di scoprire e perfezionare soluzioni informatiche che permettano di delineare con sempre maggior precisione il ruolo che hanno le mutazioni nelle regioni non codificanti del \acs{DNA}.\@ Grazie a queste nuove tecnologie, la \textsl{genomica funzionale} — area della genomica che si interessa a descrivere le relazioni che ci sono tra i componenti di un sistema biologico, come geni e proteine\,\cite{caudai2021ai} — ha avuto un forte impulso nell'approfondire le varianti non codificanti, tuttavia rimangono ancora significative lacune nella comprensione della relazione tra mutazioni genetiche ed espressione genica. L'utilizzo di tecniche di deep learning risulta quindi cruciale per continuare la ricerca in questo ambito. L'obiettivo di questo elaborato è di discutere e confrontare tre tool che utilizzano le \textsl{reti neurali convoluzionali} per predire l'effetto delle varianti non codificanti su sequenze genomiche: DeepSEA\,\cite{zhou2015predicting}, Basset\,\cite{kelley2016basset} e DeepSATA\,\cite{ma2023deepsata}.

Più precisamente, il Capitolo\,\ref{chp:biological-background} introdurrà le basi della biologia molecolare, necessarie per comprendere interamente l'importanza delle varianti non codificanti. Successivamente, nel Capitolo\,\ref{chp:neural-networks} saranno approfonditi i principi fondamentali delle reti neurali e il modo in cui le reti convoluzionali possono essere utilizzate come ottimo strumento per predire l'effetto di sequenze genomiche. Il Capitolo\,\ref{chp:CNN-non-coding-variants} invece esaminerà i dettagli implementativi di ciascuno dei tre tool, indagando principalmente sugli aspetti legati alla codifica delle sequenze, alla struttura della rete e al \textsl{dataset} utilizzato per allenare il modello. Infine nel Capitolo\,\ref{chp:discussion} si riassumono le differenze analizzate nel capitolo precedente, offrendo una visione complessiva del confronto tra i tre tool.

% https://books.google.it/books?hl=it&lr=&id=Cg4WAgAAQBAJ&oi=fnd&pg=PP1&dq=Introduction+to+cell+biology&ots=yg4LdM46O3&sig=FkW8Ei_rOFccb96Sw3A28QsHuFo&redir_esc=y#v=onepage&q=Introduction%20to%20cell%20biology&f=false

% https://books.google.it/books?hl=it&lr=&id=mXiiEAAAQBAJ&oi=fnd&pg=PP1&dq=cellular+biology&ots=8O1TrOZBXp&sig=fqZhVnT0H-I3qEu1iPk4v2nvZi8&redir_esc=y#v=onepage&q=cellular%20biology&f=false

% https://books.google.it/books?hl=it&lr=&id=z4BDRcLrekMC&oi=fnd&pg=PR19&dq=cell+nucleus+organization&ots=wiXN8JbVpC&sig=oviVqMKJqsvcYwe4X37tQM28P3Y&redir_esc=y#v=onepage&q=cell%20nucleus%20organization&f=false

% Images:
% https://www.genome.gov/genetics-glossary
