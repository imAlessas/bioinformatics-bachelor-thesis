%!TEX root = ../main.tex

\chapter{Reti neurali}\label{chp:neural-networks}
% 
Nel 1950 Alan turing pubbliuca ``Computing machinery and intewlligence'' dove viene introdotto il concetto di macchina intelligente. Viene introdotto anche il test di turing che è un test che determina se una mchhina e piu o meno capace di imitare l'uomo. Nel 1956, la onferenza di Darmont (c'è Shanno, Minsky che copnia il temrine intelligenza aertrifixcale). Gli obbbiettivi era quelli di machine translation. Rosenblat negli anni 60 si inveta il percettrone, rete neruale semplicissima. Agli anni 60 seguono due risultati allo stop dell IA, vhiamto inverno della intelligenza artificiazle. Minsky pubblica un libro, chiamato perceptrons dove viene dimostrato che il percettoprne èin grado di risolvere solo probleimi semplic, libnearmente seaprabili. Esce un report, Alpac, dove viene osservato che nemmeno la task di traduzione funziona. Dagli anni 80, viene inventata la backpropagation dove le reti neurali diventano efficaci. Qui nascono anche i sistemi esperti, che sono dei sistemi che, per esempio, si basano su delle regole logiche: qualcosa dis uper dettagliato ed un esperto inietta dentro la sua conoscienza. Nel 1996/1997 Gary kasparov sfidas DeepBLUE, progettato da IBM. La prima sfida, Kasparov vince il mach overall ma perde due partite. Nella seconda edizione del 97, deepBLUE vince tutto. Per la prima volta, un esperto viene sconmfitto da una macchina. Negli anni 2000 arriva il machine learning: apprendimento automatico. Una macchina che apprende dai dati.  Reti neurali sono un sottoinsieme del ML, il deep learning sono un sottinsieme di rete neruale (LLM sono un soittinseeme di deep NN). 

\section{Principi di base ed evoluzione}

\todo{Articoli:\,\cite{mcculloch1943logical}}


\section{Reti neurali convoluzionali}

\todo{quando fai CNN dici che ML fa cagare per questi}


\begin{comment}
    Video Enkk:         https://www.youtube.com/watch?v=QB4FR8U0N6g&t=1885s
    Deep learning:      https://www.nature.com/articles/nature14539
    Storia:             https://books.google.it/books?hl=it&lr=&id=GpYFEAAAQBAJ&oi=fnd&pg=PA393&dq=History+of+artificial+intelligence&ots=V4Y3JAo1le&sig=vCrUSwvkSHjayS99EKLAHkEa_8s&redir_esc=y#v=onepage&q=History%20of%20artificial%20intelligence&f=false
\end{comment}
