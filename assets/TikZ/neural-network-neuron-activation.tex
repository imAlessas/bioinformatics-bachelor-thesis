\colorlet{myred}{red!80!black}
\colorlet{myblue}{blue!80!black}
\colorlet{mygreen}{green!60!black}
\colorlet{myorange}{orange!70!red!60!black}
\colorlet{mydarkred}{red!30!black}
\colorlet{mydarkblue}{blue!40!black}
\colorlet{mydarkgreen}{green!30!black}
\colorlet{mylightgray}{gray!30!white}

% STYLES
\tikzset{
  >=latex, % for default LaTeX arrow head
  node/.style={thick,circle,draw=myblue,minimum size=22,inner sep=0.5,outer sep=0.6},
  node in/.style={node,fill=myblue!20,draw=myblue!30!black}, % Input nodes
  node convol/.style={node,fill=myorange!20,draw=myorange!30!black}, % Convolutional nodes
  node hidden/.style={node,fill=mylightgray!20,draw=black!70}, % Hidden nodes
  node out/.style={node,fill=myred!20,draw=myred!30!black}, % Output nodes
  connect/.style={thick,mylightgray}, % Connection lines
  connect arrow/.style={-{Latex[length=4,width=3.5]},thick,mydarkblue,shorten <=0.5,shorten >=1}, % Arrows
}
\def\nstyle{int(\lay<\Nnodlen?min(2,\lay):3)} % Map layer number to appropriate style


% NEURAL NETWORK activation
% https://www.youtube.com/watch?v=aircAruvnKk&list=PLZHQObOWTQDNU6R1_67000Dx_ZCJB-3pi&index=1
\begin{tikzpicture}[x=2.7cm,y=1.6cm]
  \message{^^JNeural network activation}
  \def\NI{5} % number of nodes in input layers
  \def\NO{4} % number of nodes in output layers
  \def\yshift{0.4} % shift last node for dots
  
  % INPUT LAYER
  \foreach \i [evaluate={\c=int(\i==\NI); \y=\NI/2-\i-\c*\yshift; \index=(\i<\NI?int(\i):"n");}]
              in {1,...,\NI}{ % loop over nodes
    \node[node hidden,outer sep=0.6] (NI-\i) at (0,\y) {$a_{\index}^{(\ell - 1)}$};
  }
  
  % OUTPUT LAYER
  \foreach \i [evaluate={\c=int(\i==\NO); \y=\NO/2-\i-\c*\yshift; \index=(\i<\NO?int(\i):"m");}]
    in {\NO,...,1}{ % loop over nodes
    \ifnum\i=1 % high-lighted node
      \node[node out]
        (NO-\i) at (1,\y) {$a_{\index}^{(\ell)}$};
      \foreach \j [evaluate={\index=(\j<\NI?int(\j):"n");}] in {1,...,\NI}{ % loop over nodes in previous layer
        \draw[connect,white,line width=1.2] (NI-\j) -- (NO-\i);
        \draw[connect, myred!50] (NI-\j) -- (NO-\i)
          node[pos=0.50] {\contour{white}{$w_{1,\index}$}};
      }
    \else % other light-colored nodes
      \node[node hidden]
        (NO-\i) at (1,\y) {$a_{\index}^{(\ell)}$};
      \foreach \j in {1,...,\NI}{ % loop over nodes in previous layer
        %\draw[connect,white,line width=1.2] (NI-\j) -- (NO-\i);
        \draw[connect] (NI-\j) -- (NO-\i);
      }
    \fi
  }
  
  % DOTS
  \path (NI-\NI) --++ (0,1+\yshift) node[midway,scale=1.2] {$\vdots$};
  \path (NO-\NO) --++ (0,1+\yshift) node[midway,scale=1.2] {$\vdots$};
  
  % EQUATIONS
  \def\agr#1{{\color{mydarkgreen}a_{#1}^{(\ell - 1)}}} % green a_i^j
  \node[below=16,right=11,mydarkblue,scale=0.95] at (NO-1)
    {$\begin{aligned} %\underset{\text{bias}}{b_1}
       &= \color{mydarkred}\sigma\left( \color{black}
            w_{1,1}\agr{1} + w_{1,2}\agr{2} + \ldots + w_{1,n}\agr{n} + b_1^{(\ell - 1)}
          \color{mydarkred}\right)\\
       &= \color{mydarkred}\sigma\left( \color{black}
            \sum_{i=1}^{n} w_{1,i}\agr{i} + b_1^{(\ell - 1)}
           \color{mydarkred}\right)
     \end{aligned}$};
  
\end{tikzpicture}