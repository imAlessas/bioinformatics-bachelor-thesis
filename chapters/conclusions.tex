%!TEX root = ../main.tex

\chapter{Conclusioni}\label{chp:conclusions}

I tre strumenti bioinformatici discussi hanno dato un grande contributo alla comprensione degli effetti delle varianti non codificanti nel genoma umano. Per migliorare le prestazioni predittive e rendere questi strumenti ancora più precisi, si consiglia di allenare i modelli con dataset più ampi, così da approfondire meglio le ricadute funzionali delle mutazioni non codificanti. Le ricerche future si focalizzeranno sulla previsione delle interazioni tra enhancer e promoter e sul loro contributo all'espressione genica, offrendo nuove prospettive sugli effetti delle varianti non codificanti.

La progettazione di tool come questi, finalizzati allo studio delle varianti non codificanti, rimane di fondamentale importanza. Nonostante il progresso raggiunto grazie a questi strumenti, il campo delle mutazioni non codificanti è ancora poco esplorato dalla comunità scientifica e necessita quindi di un costante e continuo approfondimento e miglioramento. Le malattie legate a queste mutazioni sono diffuse e ancora poco comprese. È dunque essenziale proseguire le ricerche in questo ambito, nella speranza di fare nuove scoperte che possano arricchire la nostra comprensione e portare a soluzioni innovative.


% We expect DeepSEA to help unveil the regulatory information in the vast and currently poorly understood noncoding genomic regions and contribute to understanding the potential functions of complex disease or trait-associated SNPs. The approach can be readily adapted, and likely further improved, as knowledge of functional variants increases, providing additional training data.

% With Basset, a researcher can perform a single sequencing assay in their cell type of interest and simultaneously learn that cell’s chromatin accessibility code and annotate every mutation in the genome with its influence on present accessibility and latent potential for accessibility. By leveraging large-scale public data, one can train accurate models on common computational hardware. Researchers continue to discover noncoding variants in human populations that influence phenotypes, and such annotation will be indispensable for interpreting how those variants function. As the tide of functional genomics data continues to flow, novel machine learning approaches such as deep CNNs have great power to aid this goal.
% These scores are highly predictive of the causal SNP among sets of linked variants. Importantly, Basset puts CNNs in the hands of the genome biology community, providing tools and strategies for researchers to train and analyze models on new data sets. In conjunction with genomic big data, Basset offers a promising future for understanding how the genome crafts phenotypes.

% Overall, we are confident that DeepSATA will serve as a valuable tool for annotating and prioritizing regulatory variants, thereby enhancing our understanding of their mechanisms in different tissues across diverse animal species. We believe that DeepSATA will make a significant contribution to deciphering economic traits in livestock and addressing complex diseases in humans.
% This study has presented DeepSATA, a new deep learning-based sequence analyzer that incorporates the transcription factor binding affinity for cross-species prediction. Through an analysis of the open chromatin data of pigs, chickens, cattle, humans, and mice, we demonstrated the enhanced prediction accuracy of the chromatin accessibility and the reliable cross-species prediction capabilities of DeepSATA. Furthermore, we showcased its effectiveness in analyzing pig genetic variants associated with economic traits. Overall, DeepSATA represents a valuable tool that broadens our knowledge of the potential influence of non-coding variants on complex traits across different species.
